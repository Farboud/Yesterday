\documentclass[a5paper]{book}
\usepackage[margin=2cm]{geometry}
\usepackage[utf8]{inputenc}
\usepackage[hidelinks,unicode]{hyperref}
\hypersetup{pdftitle=دیروز}
\hypersetup{pdfauthor=فربد خاتمی}
\makeatother
\usepackage{graphicx}
\graphicspath{{./Figures/}}
\usepackage{dirtytalk}
\usepackage[extrafootnotefeatures]{xepersian}
\settextfont{Yas}
\renewcommand{\baselinestretch}{1.2}
\title{\Huge{دیروز}}
\author{هاروکی موراکامی \\ مترجم: فربد خاتمی}

\begin{document}
\maketitle

این داستان از ترجمه‌ی انگلیسی فیلیپ گابریل که در شماره‌های ۹ و ۱۶ ژوئن سال ۲۰۱۴ نشریه‌ی نیویورکر چاپ شده‌است به فارسی برگردانده شده‌است.

امیدوارم که از نظر حق مؤلف، قانونی را زیر پا نگذاشته باشم.

\chapter*{دیروز}
تا جایی که می‌دانم، تنها کسی  که روی آهنگ \emph{دیروز}\LTRfootnote{Yesterday} بیتل‌ها شعر ژاپنی نوشت (و این کار را با گویش کانسای انجام داد) شخصی به نام کیتارو\LTRfootnote{Kitaru} بود. او نسخه‌ی خودش از آهنگ را در حمّام با صدای بلند آواز می‌خواند.


«دیروز


دو روز قبل از فرداست،


روز بعد از دو روز پیش است.»

و تا جایی که یادم می‌آید، شعر او اینگونه شروع می‌شد؛ البته مدّت‌هاست که به آن گوش نداده‌ام و بنابراین مطمئن نیستم که چطور ادامه پیدا می‌کرد. شعر کیتارو، از آغاز تا پایان، کاملاً بی‌معنی بود، چرندیاتی که کوچک‌ترین ارتباطی به متن اصلی ترانه نداشت. آن ملودی دوست‌داشتنی غمگین که با گویش سرزنده‌ی کانسای -که می‌توان آن را نقطه‌ی مقابل غم و اندوه دانست- همراه می‌شد، ترکیب عجیبی ایجاد می‌کرد؛ دقیقاً نقطه‌ی مقابل یک ترکیب سازنده بود. حداقل، به نظر من که اینطور می‌رسید. آن زمان فقط به آهنگ او گوش داده و سر تکان داده بودم. گرچه کلّی به آن خندیده بودم، ولی مفهوم پنهانی را در آن احساس می‌کردم.

کیتارو را اوّلین‌بار در کافی‌شاپی، واقع در نزدیکی درب اصلی دانشگاه واسِدا\LTRfootnote{Waseda}، که هردویمان در آن کار می‌کردیم ملاقات کردم. من در آشپزخانه کار می‌کردم و کیتارو پیش‌خدمت بود. ما در وقت آزادمان با یکدیگر زیاد صحبت می‌کردیم. هر دو بیست‌ساله بودیم و روزهای تولّدمان تنها یک هفته با هم اختلاف داشت.

روزی به او گفتم: «کیتارو نام خانوادگی عجیبی است.»

کیتارو با گویش غلیط کانسای خود جواب داد: «بله، همین‌طور است.»

«تیم بیسبال لوته\LTRfootnote{Lotte} هم یک پرتاب‌کننده به همین نام داشت.»

«با هم فامیل نیستیم. البته اسم خیلی مرسومی هم نیست، کسی چه می‌داند؟ شاید ارتباطی با هم داشته‌باشیم.»

آن زمان، دانشجوی سال آخر دانشکده‌ی ادبیات دانشگاه واسِدا بودم. کیتارو در امتحان ورودی دانشگاه رد شده‌بود و به کلاس آمادگی آزمون می‌رفت تا برای امتحان مجدّد آماده شود. او در واقع دوبار در امتحان رد شده‌بود، که از روی طرز رفتارش اصلاً قابل حدس نبود. زیاد درس نمی‌خواند. در وقت آزادش زیاد مطالعه می‌کرد، ولی نه کتاب‌های درسی -زندگی‌نامه جیمی هندریکس، کتاب مسائل شوگی\footnote{
\lr{Shogi}: نوعی بازی دو نفره که به شطرنج ژاپنی نیز موسوم است.}،
'جهان چطور پدید آمد؟`،
و از این قبیل کتاب‌ها. به من گفت که از خانه‌ی والدینش در اوتاوارد\LTRfootnote{Ota Ward} توکیو به کلاس آمادگی آزمون می‌رفت.

با تعجّب پرسیدم: «اوتاوارد؟ فکر می‌کردم اهل کانسای هستی.»

«به هیچ وجه. در دننچوفو\LTRfootnote{Denenchofu} به دنیا آمدم و بزرگ شدم.»

متعجّب شدم.

پرسیدم: «پس چرا با گویش کانسای حرف می‌زنی؟»

«یاد گرفتم. یک روز عزمم را جزم کردم که گویش کانسای را یاد بگیرم.»

«یاد گرفتی؟»

«بله. سخت تلاش کرم. افعال، اسامی، لهجه، همه‌چیز. درست مانند یادگیری زبان انگلیسی یا فرانسه است. برای تمرین حتّی به کانسای هم رفتم.»

پس کسانی هم بودند که گویش کانسای را مانند یک زبان خارجی یاد می‌گرفتند؟ این مسأله برایم جدید بود. این امر باعث شد تا دوباره به این نکته فکر کنم که توکیو چقدر بزرگ بود و چه چیزهایی بودند که من نمی‌دانستم.  تمام این‌ها من را به یاد کتاب 'سانشیرو\LTRfootnote{Sanshiro}` می‌انداخت، یک داستان تیپ در مورد پسری روستایی که به یک شهر بزرگ می‌آید.

کیتارو توضیح داد که: «وقتی بچّه بودم، طرفدار پروپاقرص تیم ببرهای هنشین\LTRfootnote{Hanshin Tigers} بودم.» «هر وقت در توکیو بازی داشتند، به تماشای آن‌ها می‌رفتم. امّا اگر در جمع هواداران هنشین می‌نشستم و با گویش توکیو صحبت می‌کردم، کسی کاری با من نداشت. می‌دانی؟ نمی‌توانستم جزئی از جمع آن‌ها باشم. پس به نظرم رسید که باید گویش کانسای را یاد بگیرم و برای این منظور مثل سگ تلاش کردم.»

«پس انگیزه‌ات این بود؟»

باورش برایم سخت بود.

کیتارو گفت: «درست است. ببرها تا این حد برایم مهم بودند. حالا فقط به گویش کانسای صحبت می‌کنم -در مدرسه، در خانه، و حتّی وقتی در خواب حرف می‌زنم. گویشم تقریباً بدون نقص است، اینطور فکر نمی‌کنی؟»

گفتم: «البته. مطمئن بودم که اهل کانسای هستی.»

«اگر آنطور که برای یادگیری کانسای تلاش کردم، برای آزمون ورودی دانشگاه درس می‌خواندم تا حالا دو دفعه در آزمون ورودی رد نشده‌بودم.»

درست می‌گفت. حتّی شکست‌نفسی را هم به شیوه‌ی کانسای انجام می‌داد.

پرسید: «خب. خودت اهل کجا هستی؟»

گفتم: «کانسای، نزدیک کوبه.»

«نزدیک کوبه؟ کجا؟»

جواب دادم: «آشیا
\LTRfootnote{Ashiya}».


«وای، جای قشنگی است. چرا از اوّل نگفتی؟»

توضیح دادم. وقتی مردم از من می‌پرسیدند که اهل کجا هستم و می‌گفتم آشیا، همیشه تصوّر می‌کردند که خانواده‌ی متموّلی دارم. ولی در آشیا هر نوع آدمی هست. برای مثال، خانواده‌ی خودم آنقدرها هم پول‌دار نبودند. پدرم در یک شرکت دارویی کارمی‌کرد و مادرم کتابدار بود. خانه‌ی ما کوچک بود و یک کرولای کرم‌رنگ داشتیم. بنابراین وقتی مردم از من می‌پرسیدند که اهل کجا هستم، همیشه می‌گفتم «نزدیکی کوبه»
،  تا دیگر در مورد من فرضیات پیش‌داورانه نکنند.

کیتارو گفت: «پسر، انگار که من و تو یکی هستیم». «من در دننچوفو\LTRfootnote{Denenchofu} زندگی می‌کنم -یک جای باکلاس- ولی خانه‌ی من در کهنه‌ترین بخش شهر است. خانه‌ی من خیلی هم قدیمی است. بعضی‌وقت‌ها باید به من سر بزنی. فکر می‌کنی، چی؟ اینجا هم دننچوفو است؟ امکان ندارد! ولی نگرانی در مورد این مسائل منطقی نیست، مگر نه؟ فقط یک آدرس است دیگر. من برعکس این عمل می‌کنم -مستقیم این نکته را به همه یادآور می‌شوم که در \textbf{دننچوفو} زندگی می‌کنم. طوری که انگار بگویم، خوشت آمد؟»

تحت تأثیر قرار گرفتم. و بعد از آن با هم دوست شدیم.

تا وقتی که از دبیرستان فارغ‌التحصیل شوم، فقط با گویش کانسای حرف می‌زدم. ولی گذراندن فقط یک ماه در توکیو کافی‌ بود تا به گویش استاندارد توکیو مسلّط شوم. کمی تعجّب کرده‌بودم که توانسته‌ام خودم را با شرایط موجود تطبیق دهم. شاید دوشخصیتی هستم. شاید هم حس زبان‌آموزی من از سایر مردم قوی‌تر است. در هر صورت، حالا هیچ کس باور نمی‌کند که در واقع اهل کانسای باشم.

دلیل دیگری که گویش کانسای را رها کردم این بود که می‌خواستم به یک آدم کاملاً جدید تبدیل شوم.

وقتی از کانسای به توکیو آمدم تا کالج را شروع کنم، کل مسیر را در قطارِ فشنگی به مرور هجده‌سال سنی که داشتم گذراندم و به این نتیجه رسیدم که تقریباً هر اتفاقی که برای من افتاده بود باعث شرمندگی‌ام شده‌بود. اغراق نمی‌کنم. اصلاً دلم نمی‌خواست هیچ‌کدام از اتّفاقات گذشته را به یاد داشته‌باشم -بسیار رقّت‌انگیز بود. تا آن زمان، هرچه بیشتر در مورد زندگی‌ام فکر می‌کردم، بیشتر از خودم بدم می‌آمد. مسأله این نبود که خاطرات خوب نداشتم -داشتم. تعدادی خاطره‌ی خوب. ولی وقتی در کل نگاه می‌کردی، خاطراتِ ننگین و دردناک به مراتب از دیگرْ خاطرات بیشتر بودند. وقتی به این فکر کردم که چگونه زندگی می‌کرده‌ام و نگاهم نسبت به زندگی چگونه بوده‌است، همه‌اش خیلی راکد و بی‌روح بود، به طرز متأثرکننده‌ای بی معنی و مفهوم بود.

همه‌اش مزخرفاتی بدون خلاقیّت از طبقه‌ی متوسّط جامعه بود، و دلم می‌خواست که جمعش کنم و در یک کمد انبار کنم. یا شاید همه‌اش را آتش بزنم و دودشدن و به هوا رفتن آن را تماشا کنم (گرچه هیچ تصوّری نداشتم که چه دودی از آن متصاعد می‌شد). در هر صورت، می‌خواستم از شرّ همه‌ی آن‌ها خلاص شوم و زندگی جدیدی را در قالب یک آدم جدید در توکیو شروع کنم. رهاکردن گویش کانسای یک روش عملی (و همچنین نمادین) برای نیل به این هدف بود. چون، در بررسی نهایی، زبانی که به آن صحبت می‌کنیم بخشی از شخصیّت ما به عنوان انسان است. حداقل در سن هجده‌سالگی اینطور به نظرم می‌رسید.

کیتارو پرسید: «ننگین؟ چه چیزی آنقدر باعث شرمندگی بود؟»

«خودت چه فکر می‌کنی؟»

«با آشنایانت خوب تا نمی‌کردی؟»

گفتم: «با خانواده‌ام خوب کنار می‌آمدم.» «ولی همچنان باعث شرمندگی‌ام بود. در کنار دیگران بودن باعث احساس شرمندگی می‌شد.»

کیتارو گفت: «تو آدم عجیبی هستی. می‌دانستی؟» «در کنار اقوام بودن کجایش باعث احساس شرمندگی می‌شود؟ من که در کنار آشنایانم اوقات خوشی دارم.»

نمی‌توانستم کاملاً توضیح دهم. داشتن یک کرولای کِرِم‌رنگ کجایش بد است؟ نمی‌توانستم بگویم. والدینم علاقه‌ای نداشتند برای ظواهر پول خرج کنند، همین.

«والدین من همیشه مرا تحت نظر دارند، چون به قدر کافی درس نمی‌خوانم. از درس‌خواندن متنفّرم، ولی چه می‌شود کرد؟ کار والدین همین است دیگر. نباید خیلی سخت بگیری، می‌دانی؟»

گفتم: «خیلی سخت‌گیر نیستی، مگرنه؟»

کیتارو پرسید: «دوست‌دختر داری؟»

«فعلاً نه.»

«ولی قبلاً داشتی؟»

«تا کمی قبل.»

«بهم زدید؟»

گفتم: «بله، همینطور است».

«چرا بهم زدید؟»

«داستانش خیلی مفصّل است، نمی‌خواهم وارد آن شوم.»

«کاملاً رهایت کرد که بروی؟»

سرم را تکان دادم. «نه. نه کاملاً.»

«به همین دلیل بهم زدید؟»

کمی فکر کردم. «این هم بخشی از ماجراست.»

«گذاشت با هم صمیمی شوید؟»

«نزدیکش بود.»

«دقیقاً تا کجا پیش رفته بودی؟»

گفتم: «نمی‌خواهم در موردش صحبت کنم.»

«این هم یکی از آن موجبات شرمندگی است که گفته‌بودی؟»

گفتم: «بله».

کیتارو گفت: «پسر، عجب زندگی پیچیده‌ای داری».

اوّلین باری که شنیدم کیتارو \emph{دیروز} را با آن اشعار دیوانه‌وارش بخواند، زمانی بود که داشت در خانه‌اش در دننچوفو (که برخلاف توصیفاتش، یک خانه‌ی قدیمی در یک منطقه‌ی کهنه‌ی شهر نبود، بلکه یک خانه‌ی معمولی در در یک محلّه‌ی معمولی بود، خانه‌ای که از منزل من در آشیا کمی قدیمی‌تر ولی بزرگ‌تر بود ولی از هیچ جهتی نسبت به آن برتری نداشت -ضمناً ماشین جلوی گاراژ یک گلف مدل جدید به رنگ سرمه‌ای بود.) حمّام می‌کرد. کیتارو هر وقت به منزل بر می‌گشت، هر چیز که در دست داشت را رها می‌کرد و می‌پرید داخل حمّام. و زمانی که وارد وان می‌شد، تا ابد در آن می‌ماند. بنابراین من اغلب یک چهارپایه‌ی گرد را به اتاق مجاور می‌کشیدم، روی آن می‌نشستم و از پشت یک در کشویی که لای آن حدوداً دو سانتیمتر باز بود با او حرف می‌زدم. این تنها راهی بود که می‌توانستم از غرغر بی‌پایان مادرش خلاص شوم (اغلب در مورد پسر عجیبش و اینکه باید بیشتر درس بخواند شکایت می‌کرد).

به او گفتم: «این شعر هیچ معنایی ندارد». «اینطور به نظر می‌رسد که داری آهنگ \emph{دیروز} را مسخره می‌کنی.»

«زرنگی نکن. آهنگ را مسخره نمی‌کنم. حتّی اگر هم قصدم مسخره‌کردن بود، نباید فراموش کنی که جان اشعار بی‌معنی و بازی با کلمات را دوست داشت. مگر نه؟»

«ولی پُل کسی بود که شعر و آهنگ \emph{دیروز} را نوشت.»

«مطمئنی؟»

گفتم: «البته.» «پُل آهنگ را نوشت و خودش هم آن را در استودیو با یک گیتار اجرا و ضبط کرد. یک قطعه‌ی جهاربخشی بعداً به آن اضافه شد، ولی بقیه‌ی بیتل‌ها اصلاً در آن نقشی نداشتند. فکر می‌کردند که به عنوان یکی از آهنگ‌های بیتل‌ها زیادی ضعیف است.»

«جدّاً؟ من خیلی به این اطّلاعات خاص مطّلع نیستم.»

گفتم: «این که اطّلاعات خاص نیست.» «مسأله‌ای است که همه می‌دانند.»

صدای کیتارو از میان ابری از بخار گفت: «چه کسی اهمیّت می‌دهد؟ اینها فقط جزيیّات هستند.» «من دارم در حمّام خانه‌ی خودم آواز می‌خوانم. قرار نیست که آن را ضبط کنم که. نه حق مؤلف آهنگ را زیر پا گذاشته‌ام و نه به احدی آسیب می‌رسانم. تو هم حق اعتراض نداری.»

و می‌زد زیر آواز، صدایش واضح و بلند بود. نت‌های بالا را به خصوص خیلی خوب می‌خواند. می‌توانستم صدای شلپ شلپ آب را هم که برای همراهی با آهنگ ایجاد می‌کرد بشنوم. من هم احتمالاً باید همراه او می‌خوانم تا تشویقش کنم، ولی نمی‌توانستم خودم را به این کار راضی کنم. همان طور نشستن و صحبت از پشت یک در شیشه‌ای برای همراهی با او، در حالی که خودش یک ساعت در حمّام خیس می‌خورد اصلاً مفرّح نبود.

پرسیدم: «چطور می‌توانی اینقدر طولانی‌مدّت در حمّام خیس بخوری؟» «بدنت ورم نمی‌کند؟»

کیتارو گفت: «وقتی در آب خیس می‌خورم، انواع و اقسام فکرهای خوب به ذهنم می‌آیند.»

«مثلاً مثل شعر \emph{دیروز}؟»

کیتارو گفت: «خب این هم یکی از آن افکار بود.»

پرسیدم: «بهتر نیست به جای این همه خیس خوردن در حمّام برای آزمون ورودی دانشگاه درس بخوانی؟»

«خدایا، تو هم حال آدم را می‌گیری. مادرم هم دقیقاً همین را می‌گوید. برای اینکه به من خِرَدورزی بیاموزی کمی جوان نیستی؟»

«دو سال است که داری درس می‌خوانی، از این کار خسته نشده‌ای؟»

«چرا. البته که من هم می‌خواهم هرچه زودتر به کالج بروم.»

«پس چرا بیشتر درس نمی‌خوانی؟»

همانطور که کلمات را پشت هم می‌چید گفت: «خب، اگر می‌توانستم بیشتر درس بخوانم که تا الآن بیشتر درس خوانده بودم.»

گفتم: «کالج خیلی خسته‌کننده‌است.» «وقتی به کالج رفتم حسابی ناامید شدم. ولی وارد کالج نشدن بیشتر خسته‌کننده است.»

کیتارو گفت: «قبول.» «جوابی برای این حرفت ندارم.»

«پس چرا درس نمی‌خوانی؟»

گفت: «کمبود انگیزه.»

گفتم: «انگیزه؟» «اینکه بتوانی با دوست‌دخترت قرار بگذاری انگیزه‌ی خوبی نیست؟»

دختری بود که کیتارو از زمان دبستان او را می‌شناخت. می‌شد گفت، دوست‌دختر دوران کودکی. آن‌ها در یک دبیرستان درس می‌خواندند، ولی برخلاف کیتارو، دختر مستقیماً بعد از دبیرستان به دانشگاه سوفیا\LTRfootnote{Sophia} رفته‌بود.  در رشته‌ی زبان فرانسه درس می‌خواند و به کلوپ تنیس پیوسته بود. کیتارو عکس او را به من نشان داده‌بود؛ و از زیبایی متحیّر کننده بود. ژست زیبایی داشت و صورتش سرزنده بود. ولی آن دو، این روزها زیاد همدیگر را نمی‌دیدند. تصمیم گرفته‌بودند که بهتر است تا وقتی کیتارو در امتحان ورودی قبول نشده‌است با هم قرار نگذارند، تا کیتارو بتواند روی درس‌هایش تمرکز کند. کیتارو خودش چنین پیشنهادی را داده‌بود. دختر هم گفته‌بود: «باشد، اگر این چیزیست که می‌خواهی، من هم حرفی ندارم.» پای تلفن زیاد با یکدیگر صحبت می‌کردند ولی حداکثر هفته‌ای یکبار با هم ملاقات داشتند، ملاقات‌هایی که بیشتر شبیه مصاحبه بود تا یک قرار معمولی.  با هم چای می‌خوردند و در مورد کارهایی که انجام داده‌بودند به همدیگر می‌گفتند. دست همدیگر را می‌گرفتند و بوسه‌ای کوتاه رد و بدل می‌کردند، قرارشان در همین حد بود.

کیتارو از دسته‌ی آدم‌های زیبا نبود، ولی به قدر کافی خوش‌چهره بود. لاغر بود، و لباس‌ها و مدل مویش ساده و شیک بودند. مادامی که شروع به صحبت نمی‌کرد، فکر می‌کردی که یک پسر شهری حسّاس و نازپرورده است. تنها نقص احتمالی‌اش صورتش بود، که کمی بیش از حد لاغر و ظریف بود و این حس را به آدم منتقل می‌کرد که او یا شخصیّت قدرتمندی ندارد و یا دمدمی‌مزاج است. ولی لحظه‌ای که دهانش را باز می‌کرد، تمام جنبه‌های مثبتش مانند یک قلعه‌ی شنی زیر پنجه‌های یک سگ لابرادور فرو می‌ریخت. مردم از گویش کانسای او، که اتّفاقاً به خوبی ادا می‌شد، منزجر می‌شدند و اگر این گویش به تنهایی کافی نبود، صدایش نیز بلند و نافذ بود. عدم تطابق صدا و ظاهر او بیش از حد بود؛ حتّی اوایل برای من هم تحملّش دشوار بود.

کیتارو روز بعد از من پرسید: «هی تانیمورا\LTRfootnote{Tanimura}، بدون دوست‌دختر تنها نیستی؟»

گفتم: «انکار نمی‌کنم.»

«پس نظرت چیست که با دوست‌دختر من بیرون بروی؟»

متوجّه منظورش نشدم. «منظورت چیست که با او بیرون بروم؟»

«دختر خیلی خوبی است. زیبا، صادق و باهوش است. اگر با او بیرون بروی پشیمان نمی‌شوی. تضمین می‌کنم.»

گفتم: «مطمئنم که پشیمان نمی‌شوم.» «ولی چرا باید با دوست‌دختر تو بیرون بروم؟ منطقی نیست.»

کیتارو گفت: «چون تو پسر خوبی هستی.» «در غیر این صورت چنین پیشنهادی نمی‌دادم. اریکا\LTRfootnote{Erika} و من تا الآن تقریباً همه‌ی زندگی‌مان را در کنار هم گذرانده‌ایم. طبیعتاً تبدیل به یک زوج شدیم و همه‌ی آشنایانمان از این موضوع راضی بودند. دوستانمان، والدینمان و معلّمینمان. یک زوج صمیمی کوچک که همیشه با هم هستند.»

کیتارو دست‌هایش را به هم فشرد تا نشان دهد.

«اگر هردو مستقیم به کالج رفته‌بودیم، زندگی هردویمان اکنون گرم بود و به نتیجه رسیده‌بود، ولی من در آزمون ورودی حسابی خراب‌کاری کردم و نتیجه این شد. دقیقاً نمی‌دانم چرا، ولی بعد از آن اوضاع مدام بدتر شد. کسی را برای آن سرزنش نمی‌کنم -همه‌اش تقصیر خودم است.»

در سکوت به حرف‌هایش گوش دادم.

کیتارو گفت: «بنابراین خودم را دو نیم کردم.» دستانش را از هم باز کرد.

پرسیدم: «چطور؟»

لحظه‌ای به کف دستانش خیره شد و سپس ادامه داد. «منظورم این است که بخشی از وجودم انگار نگران است، متوجّه می‌شوی؟ منظورم این است که به کلاس لعنتی آمادگی آزمون می‌روم، و برای آزمون لعنتی درس می‌خوانم، در حالی که اریکا در کالج مشغول ورزش و تفریح است. تنیس بازی می‌کند و به کارهایش می‌رسد. تا جایی که می‌دانم دوستان تازه پیدا کرده و احتمالاً  با شخص دیگری هم بیرون می‌رود. وقتی به همه‌ی این‌ها فکر می‌کنم، به نظرم می‌رسد که جا مانده‌ام. انگار ذهنم در مِه گیر کرده‌باشد. متوجّه منظورم هستی؟»

گفتم: «به گمانم.»

«ولی بخش دیگری از وجودم انگار -آسوده‌ است؟ اگر مثل سابق ادامه می‌دادیم، بدون هیچ مشکلی، زوج خوبی که به آرامی در زندگی به پیش می‌روند، زندگی‌مان اینطور می‌شد \ldots 
از کالج فارغ‌التّحصیل می‌شدیم، ازدواج می‌کردیم و می‌شدیم آن زوج فوق‌العاده و همه برایمان خوشحال بودند، مثل همه صاحب دو فرزند می‌شدیم، آن‌ها را به مدرسه‌ی ابتدایی دننچوفو می‌فرستادیم، یکشنبه‌ها به کنار رودخانه‌ی تاما\LTRfootnote{Tama} می‌رفتیم، زندگی ادامه پیدا می‌کرد
\footnote{
\lr{Ob-la-di, Ob-la-da}: اشاره دارد به آهنگی به همین نام از بیتل‌ها که در مورد زندگی خوش یک زوج است.}
و \ldots نمی‌گویم که این طور زندگی بد است. ولی می‌خواهم بدانم که ،می‌دانی، زندگی باید اینقدر ساده باشد، اینقدر راحت باشد؟ شاید بهتر باشد که هر دویمان مدّتی راه مجزای خودمان را دنبال کنیم، و اگر به این نتیجه رسیدیم که بدون یکدیگر نمی‌توانیم ادامه دهیم، دوباره پیش هم برگردیم.»

«یعنی می‌گویی این که همه‌چیز سرراست و راحت باشد یک مشکل است؟ منظورت این است؟»

«تقریباً همینطور است.»

پرسیدم: «ولی چرا من باید با دوست‌دخترت بیرون بروم؟»

«فکر کردم، اگر قرار باشد با پسر دیگری بیرون برود، بهتر است که آن پسر تو باشی. چون تو را می‌شناسم. و می‌توانی هر از گاهی من را در جریان اخبار و وقایع جدید بگذاری.»

حرف‌هایش اصلاً به نظرم منطقی نبود، گرچه باید اعتراف کنم که از ایده‌ی ملاقات اریکا خوشحال شدم. همچنین می‌خواستم بدانم که دختر زیبایی مثل اریکا چرا با شخص عجیبی مثل کیتارو بیرون می‌رفت. من همیشه کنار مردم کمی خجالتی بودم، ولی هیچ وقت فاقد حس کنجکاوی نبوده‌ام.

پرسیدم: «با او تا کجا پیش رفته‌ای؟»

کیتارو گفت: «منظورت این است که رابطه داشته‌ایم یا نه؟»

«بله. تا آخر راه رفته‌ای؟»

کیتارو سرش را به نشانه‌ی نفی تکان داد. «نمی‌توانستم، متوجّهی؟ از زمانی که بچّه بود او را می‌شناختم، و کمی باعث شرمساری‌ام است که، می‌دانی، که جوری رفتار کنم که انگار تازه با هم آشنا شده‌ایم، که لباس‌هایش را در بیاورم، که او را ناز و نوازش کنم، یا هرچه. اگر دختر دیگری بود، فکر نمی‌کنم که مشکلی داشتم، ولی لمس لباس‌زیر او، حتّی فکر کردن به آن -نمی‌دانم- به نظر کار درستی نمی‌رسد. متوجّهی؟»

متوجّه نبودم.

کیتارو گفت: «نمی‌توانم خوب برایت توضیح دهم.» «درست مثل زمانی است که داری با خودت خلوت می‌کنی، دختری که می‌شناسی را مجسّم می‌کنی، درست می‌گویم؟»

گفتم: «به گمانم.»

«ولی من نمی‌توانم اریکا را مجسّم کنم. انگار که کار اشتباهی باشد، می‌دانی؟ پس وقتی می‌خواهم با خودم خلوت کنم به دختر دیگری فکر می‌کنم. کسی که آنقدرها هم دوستش ندارم. نظرت چیست؟»

به حرف‌هایش فکر کردم ولی به نتیجه نرسیدم. این که دیگران موقع خلوت با خود چه عاداتی دارند به من مربوط نبود. خودم هم رفتارهایی داشتم که درکشان نمی‌کردم.

کیتارو گفت:«به هر نحو، بیا هر سه نفرمان یکبار دور هم جمع شویم.» «بعد می‌توانی در مورد پیشنهادم فکر کنی.»

هر سه نفرمان -من، کیتارو و دوست‌دخترش که اسم کاملش اریکا کوریتانی\LTRfootnote{Kuritani} بود- بعد از ظهر یک روز یکشنبه در کافی‌شاپی در نزدیکی ایستگاه دننچوفو با یکدیگر ملاقات کردیم. اریکا تقریباً هم‌قد کیتارو بود، پوستش به زیبایی برنزه شده‌بود و یک بلوز سفید اتوکشیده و یک دامن کوتاه سرمه‌ای به تن داشت. درست مثل یک مدل بی‌نقص از یک دختر دانشجوی محترم و بالاشهری بود. درست همانند تصویرش زیبا بود، ولی آنچه که شخصاً مرا مجذوب خودش کرد، به جای ظاهرش، نوعی انرژی و سرزندگی بود که از وجودش ساطع می‌شد. درست برعکس کیتارو بود، که در مقایسه رنگ‌پریده به نظر می‌رسید.

اریکا به من گفت: «خیلی خوشحالم که آقای آکی‌\LTRfootnote{Aki-kun} دوست پیدا کرده‌است.» اسم کوچک کیتارو آکیوشی\LTRfootnote{Akiyoshi} بود. اریکا تنها آدمی بود که می‌توانست او را آکی خطاب کند.

کیتارو گفت: «اغراق نکن. من دوستان زیادی دارم.»

اریکا گفت: «نه، نداری.» «شخصی مانند تو نمی‌تواند دوست پیدا کند. تو در توکیو به دنیا آمده‌ای، با این وجود، فقط به گویش کانسای صحبت می‌کنی، و هر بار که دهانت را باز می‌کنی فقط جملات آزاردهنده در مورد ببرهای هنشین یا حرکت‌های شوگی از آن بیرون می‌آید. هیچ رقمه امکان ندارد که آب آدم عجیبی مثل تو با مردم عادی در یک جوی برود.»

«اگر اینطور فکر می‌کنی، این بابا هم نسبتاً عجیب است.» کیتارو به من اشاره کرد. «اهل آشیا است، ولی فقط به گویش توکیو حرف می‌زند.»

اریکا گفت: «این مرسوم‌تر است.» «حداقل از برعکسش که خیلی مرسوم‌تر است.»

کیتارو گفت: «صبر کن ببینم، این تبعیض فرهنگی است.» «همه‌ی فرهنگ‌ها با یکدیگر برابرند، متوجّهی؟ گویش توکیو نسیت به گویش کانسای برتری ندارد.»

اریکا گفت: «شاید همه با هم برابر باشند.» «ولی از زمان قانون اصلاح گفتار میجی\LTRfootnote{Meiji} نحوه‌ی صحبت مردم توکیو به عنوان استاندارد برای گفتار ژاپنی انتخاب شده‌است. منظورم این است که تا کنون کسی  کتاب \emph{فرنی و زویی}\footnote{
\lr{Franny and Zooey}: کتابی از جی دی سلینجر}
را به گویش کانسای ترجمه کرده‌است؟»

کیتارو گفت: «اگر کسی این کار را می‌کرد، حتماً آن را می‌خریدم.»

فکر کردم من هم همینطور، ولی ساکت ماندم.

اریکا کوریتانی خردمندانه موضوع صحبت را عوض کرد تا بحث عمیق‌تر نشود.

به من رو کرد و گفت: «در کلوپ تنیس هم یک دختر هست که اهل آشیاست.» «اسمش اریکو ساکورای\LTRfootnote{Eriko Sakurai} است. احیاناً او را می‌شناسی؟»

گفتم: «بله.» اریکو ساکورای یک دختر لاغر و قدبلند بود که والدینش یک زمین گلف بزرگ را اداره می‌کردند. دختری بود مغرور با سینه‌های کوچک که بینی‌اش شکل بامزه‌ای داشت و شخصیّتش خیلی فوق‌العاده نبود. تنیس از کارهایی بود که همیشه در آن مهارت داشت. حتّی اگر دیگر هیچ‌وقت نمی‌دیدمش، باز هم برای دیدنش زود بود.

کیتارو به اریکا گفت: «پسر خوبی است و فعلاً دوست‌دختر ندارد.» «قیافه‌اش بدک نیست، اخلاق خوبی دارد و خیلی با معلومات است. همان‌طور که می‌بینی تمیز و مرتّب است و مرض لاعلاج هم ندارد. به نظر من که یک مرد جوان با آینده‌‌ای روشن است.»

اریکا گفت: «باشد.» «دخترهای نازی هستند که تازه به کلوپ ملحق شده‌اند و خوشحال می‌شوم یکی از آن‌ها را به او معرّفی کنم.»

کیتارو گفت: «نه، منظورم این نیست.» «می‌توانی خودت با او بیرون بروی؟ من هنوز به کالج نرفته‌ام و نمی‌توانم آنطور که می‌خواهم با تو بیرون بروم. به جای من، می‌توانی با \emph{او} بیرون بروی. در این صورت لازم نیست که من هم نگران شوم.»

اریکا پرسید: «منظورت چیست که لازم نیست نگران شوی؟»

«منظورم این است که، خب، هردوی شما را می‌شناسم و اگر با او بیرون بروی، حس بهتری دارم تا اینکه با شخص دیگری که هیچ‌وقت او را ندیده‌ام بیرون بروی.»

اریکا انگار که آنچه می‌بیند را باور نکرده‌باشد، به کیتارو خیره شد. بالاخره گفت: «منظورت این است که اشکالی ندارد با شخص دیگری بیرون بروم، به شرطی که آن شخص آقای تانیمورا باشد؟ جداّ داری پیشنهاد می‌کنی که باهم بیرون برویم، سرِ قرار؟»

«هی، آنقدرها هم فکر بدی نیست، مگر نه؟ نکند که همین الآن هم با پسر دیگری بیرون می‌روی؟»

اریکا به آرامی گفت: «نه، شخص دیگری نیست.»

«پس چرا با او بیرون نمی‌روی؟ به چشم تبادل فرهنگی به آن نگاه کن.»

اریکا تکرار کرد: «تبادل فرهنگی.» سپس به من نگاه کرد.

به نظر نمی‌رسید که جملات من به وضع موجود کمکی کند، بنابراین ساکت ماندم. قاشق چای‌خوری را در دستم گرفتم و به طرّاحی آن دقّت کردم، درست مانند کارشناس موزه‌ای که یک شئ باستانی از یک معبد مصری را مورد بررسی قرار دهد.

از کیتارو پرسید: «تبادل فرهنگی؟ این دیگر یعنی چه؟»

«یعنی، اضافه کردن یک دیدگاه جدید شاید برایمان بد نباشد\ldots»

«تعریف تو از تبادل فرهنگی این است؟»

«خب منظورم این است که\ldots»

اریکا کوریتانی قاطعانه گفت: «خُب.» اگر مدادی در نزدیکی‌ام بود، ممکن بود آن را بردارم و دو نیم کنم. «اگر فکر می‌کنی که باید این کار را انجام دهیم، آقا آکی، باشد. بیا تبادل فرهنگی کنیم.»

جرعه‌ای از چایش را نوشید، فنجان را با نعلبکی برگرداند، به من رو کرد و لبخند زد. «چون آکی پیشنهاد داده‌است، بیا با هم قرار بگذاریم آقای تانیمورا. باید جالب باشد. کِی وقت آزاد داری؟»

نمی‌توانستم صحبت کنم. عدم توانایی انتخاب کلمات درست در لحظات بحرانی یکی از مشکلات عدیده‌ی من بود.

اریکا سالنامه‌ای با جلد چرمی قرمز از کیفش بیرون درآورد، آن را باز کرد و برنامه‌اش را مورد بررسی قرار داد. پرسید: «شنبه‌ی آینده چطور است؟»

گفتم: «برنامه‌ای ندارم.»

«پس شد برای شنبه. کجا برویم؟»

کیتارو به او گفت: «عاشق فیلم است.» «رویایش این است که روزی فیلم‌نامه بنویسد.»

«پس به سینما می‌رویم. چه جور فیلمی ببینیم؟ این را می‌گذارم بر عهده‌ی تو، آقای تانیمورا. از فیلم‌های ترسناک خوشم نمی‌آید، ولی به جز آن با هر فیلم دیگری مشکلی ندارم.»

کیتارو به من گفت: «مثل یک گربه ترسو است.» «وقتی بچّه بودیم و در شهربازی کوراکوئن\LTRfootnote{Korakuen} به تونل وحشت می‌رفتیم، مجبور بود دست من را بگیرد و -»

اریکا حرف او را قطع کرد: «بعد از فیلم هم بیا یک غذای حسابی با هم نوش جان کنیم.»  او شماره‌ی تلفنش را روی برگی از دفترچه‌اش نوشت و به من داد. «وقتی در مورد زمان و مکان تصمیم گرفتی به من تلفن می‌کنی؟»

آن زمان تلفن نداشتم (قضیه مربوط به زمانی قبل از آن است که حتّی ایده‌ی تلفن‌های همراه به ذهن کسی خطور کرده باشد)، پس شماره‌ی کافی‌شاپی که من و کیتارو در آن کار می‌کردیم را به او دادم. به ساعتم نگاهی انداختم.

تا جایی که در توانم بود با گشاده‌رویی گفتم: «ببخشید، ولی باید بروم.» «تا فردا باید گزارشی را کامل کنم.»

کیتارو گفت: «می‌تواند بماند برای بعد؟» «تازه رسیده‌ایم. چرا کمی بیشتر نمی‌مانی تا بیشتر حرف بزنیم؟ سر نبش خیابان یک نودل‌فروشی خوب است.»

اریکا نظری نداد. پول قهره‌ام را روی میز گذاشتم و بلند شدم. توضیح دادم که: «گزارش مهمی است.» «پس واقعاً نمی‌توانم آن را به تعویق بیاندازم.» ولی در واقع اصلاً مسأله‌ی مهمی نبود.

به اریکا گفتم: «فردا یا پس‌فردا زنگ می‌زنم.»

با لبخند زیبایی به لب جواب داد: «منتظر هستم.» لبخندی بود که، حداقل به نظر من، زیادی خوب بود که بتواند واقعی باشد.

کافی‌شاپ را ترک کردم و همان‌طور که به سمت ایستگاه قدم می‌زدم در فکر این بودم که داشتم چه غلطی می‌کردم. تفکّر در مورد نتیجه‌ی کارها -بعد از اینکه در مورد همه‌چیز تصمیم‌گیری شده‌بود- یکی دیگر از مشکلات دیرینه‌ی من بود.

آن شنبه، اریکا و من در شیبویا\LTRfootnote{Shibuya} ملاقات کردیم و یکی از فیلم‌های وودی آلن که داستان آن در نیو‌یورک اتّفاق می‌افتاد را تماشا کردیم. مطمئن بودم که کیتارو قبلاً او را به تماشای چنین فیلمی نبرده‌است. خوشبختانه فیلم خوبی بود و زمانی که سینما را ترک می‌کردیم هردو سر کِیف بودیم.

در هوای گرگ و میش کمی در خیابان‌ها قدم زدیم، بعد به یک رستوران ایتالیایی کوچک در ساکوراگائوکا\LTRfootnote{Sakuragaoka} رفتیم تا پیتزا و کیانتی\footnote{
\lr{Chianti}: نوعی شراب قرمز ایتالیایی.} 
بخوریم. یک رستوران معمولی با قیمت‌های معقول بود که نور کمی داشت و روی میزهایش شمع روشن کرده بودند. (آن زمان در اغلب رستوران‌های ایتالیایی روی میزها شمع می‌گذاشتند و از رومیزی‌های چهارخانه‌ی کتانی استفاده می‌کردند.) 

در مورد همه‌چیز صحبت کردیم، از آن مکالماتی که از دو دانشجوی سال آخر در قرار اوّلشان انتظار می‌رود (البته با فرض اینکه می‌شد این را یک قرار دانست). در مورد فیلمی که دیده‌بودیم صحبت کردیم، در مورد زندگی‌ دانشجویی و  سرگرمی‌هایمان. بیشتر از آنچه انتظار داشتم از مصاحبت یکدیگر لذّت بردیم و حتّی چند بار هم بلند خندیدیم. نمی‌خواهم فخرفروشی کنم، ولی استعدادی ذاتی در خنداندن دخترها دارم.

اریکا از من پرسید: «از آکی شنیدم که چند وقت قبل با دوست‌دختر دوران دبیرستانت به هم زدی؟»

جواب دادم: «بله.» «تقریباً سه سال با هم بیرون می‌رفتیم، ولی متأسفانه اوضاع خوب پیش نرفت.»

«آکی گفت که دلیل خوب پیش نرفتن اوضاع سکس بود. که او -چطور بگویم؟- آنچه می‌خواستی را به تو نمی‌داد؟»

«این هم بخشی از ماجرا بود. ولی نه همه‌اش. فکر می کنم اگر واقعاً دوستش داشتم می‌توانستم صبر کنم. منظورم این است که اگر مطمئن بودم که دوستش دارم. ولی مطمئن نبودم.»

اریکا سر تکان داد.

گفتم: «حتّی اگر تا آخر راه هم می‌رفتیم، اوضاع احتمالاً همینطور تمام می‌شد.» «فکر می‌کنم غیر قابل اجتناب بود.»

پرسید: «برایت سخت است؟»

«چه \emph{چیزی} سخت است؟»

«اینکه بعد از جزئی از یک زوج بودن، ناگهان تنها شوی.»

صادقانه گفتم: «گاهی اوقات.»

«ولی شاید گذر از تجربه‌ی سختی و بی‌کسی در جوانی لازم باشد؟ شاید این بخشی از فرایند بزرگ‌شدن است؟»

«اینطور فکر می‌کنی؟»

«همانطور که تحمّل زمستان‌های سخت باعث قوی‌تر شدن درخت می‌شود و تعداد دوایر عمر درخت را افزایش می‌دهد.»

سعی کردم دوایر عمر خودم را مجسّم کنم. ولی تنها چیزی که به ذهنم رسید تصویر بقایای یک تکّه کیک \emph{بامکوچن}\LTRfootnote{
Baumkuchen}
بود. از آن نوعی که لایه‌هایش مثل دوایر درون درخت است.

گفتم: «قبول دارم که مردم به چنین دورانی در زندگی نیاز دارند.» «حتّی بهتر است که بدانند روزی همه‌چیز تمام می‌شود.»

لبخند زد. «نگران نباش. مطمئن هستم که به زودی با شخص مناسبی آشنا می‌شوی.»

گفتم: «امیدوارم.»

اریکا همان‌طور که من پیتزایم را می‌خوردم ثابت نشست.

«آقای تانیمورا، می‌خواستم در مورد چیزی از شما مشورت بگیرم. اشکالی ندارد؟»

گفتم: «البته.» این هم یکی دیگر از مشکلاتی بود که مجبور بودم با آن دست و پنجه نرم کنم: آدم‌هایی که تازه ملاقات کرده بودم در مورد موضوع مهمی به پند و اندرز من نیاز داشتند. و تقریباً مطمئن بودم که اریکا نظر من را در مورد یک موضوع ناخوشایند می‌خواهد.

شروع کرد: «گیج شده‌ام.»

چشمانش درست مانند چشمان گربه‌ای که به دنبال چیزی باشد به اطراف حرکت می‌کرد.

«مطمئن هستم که این موضوع را می‌دانی، ولی آکی دومین سالی است که برای آزمون ورودی مطالعه می‌کند، به ندرت درس می‌خواند. بسیاری از اوقات در کلاس‌های آمادگی آزمون هم حاضر نمی‌شود. مطمئنم که سال بعد هم در آزمون موفّق نخواهد شد. اگر هدفش یکی از دانشگاه‌های سطح پایین‌تر بود، می‌توانست جایی قبول شود، ولی هدفش دانشگاه واسِدا است. نه به حرف من گوش می‌دهد و نه به حرف والدینش. تمام فکر و ذکرش شده است این \ldots ولی اگر واقعاً اینطور احساس کند، باید سخت تلاش کند تا در امتحان واسِدا موفّق شود، ولی نمی‌کند.»

«چرا بیشتر درس نمی‌خواند؟»

اریکا گفت: «واقعاً معتقد است که اگر شانس با او یار باشد در امتحان قبول می‌شود.» «فکر می‌کند که درس‌خواندن وقت تلف کردن است.» آه کشید و ادامه داد: «در مدرسه‌ی ابتدایی از نظر تحصیلی نفر اوّل کلاس بود. ولی به دبیرستان که رسید، نمره‌هایش شروع کرد به افت کردن. زمان بچّگی یک اعجوبه بود -شخصیتش مناسب درس‌خواندنِ یکنواختِ روزانه نیست. ترجیح می‌دهد برود کارهای دیوانه‌وار خودش را انجام دهد. من دقیقاً برعکس او هستم. آنقدرها هم باهوش نیستم، ولی کمر همّت می‌بندم و کارها را به سرانجام می‌رسانم.»

من خودم خیلی سخت درس نخوانده بودم و همان بار اوّل هم به دانشگاه رفته بودم. شاید شانس با من یار بود.

ادامه داد: «واقعاً به آکی افتخار می‌کنم.» «ویژگی‌های فوق‌العاده‌ی زیادی دارد. ولی گاهی اوقات برایم دشوار است که طرز فکر او را تحمّل کنم. حالا با گویش کانسای هم صحبت می‌کند. چرا کسی که در توکیو به دنیا آمده و بزرگ شده باید خود را به دردسر یادگیری گویش کانسای بیاندازد و دائم هم به این گویش صحبت کند؟ نمی‌فهمم، واقعاً نمی‌فهمم. اوّلش فکر می‌کردم دارد شوخی می‌کند، ولی نمی‌کرد. خیلی هم جدّی بود.»

گفتم: «فکر می‌کنم می‌خواهد شخصیّت متفاوتی داشته‌باشد، می‌خواهد شخصی متفاوت با کسی که تاکنون بوده‌است، باشد.»

«به همین دلیل است که فقط به گویش کانسای صحبت می‌کند؟»

«با تو موافقم که روش او کمی افراطی است.»

اریکا برشی از پیتزا برداشت و به اندازه‌ی یک تمبر پستی از آن را گاز زد. قبل از اینکه ادامه دهد، آن را غرق در تفکّر جوید.

«آقای تانیمورا، این‌ها را از شما می‌پرسم چون کسی دیگری را ندارم که با او مشورت کنم. ناراحت که نمی‌شوید؟»

گفتم: «البته که نه.» چه چیز دیگری می‌توانستم بگویم؟

گفت: «به عنوان یک قاعده‌ی کلّی، وقتی یک پسر و دختر برای مدّتی طولانی بیرون می‌روند و با همدیگر خوب آشنا می‌شوند، پسر به دختر از نظر جنسی علاقه نشان می‌دهد. درست است؟»

«به عنوان یک قاعده‌ی کلّی می‌توان اینطور گفت، بله.»

«اگر همدیگر را ببوسند، پسر می‌خواهد که بیشتر به پیش بروند؟»

«معمولاً، همین‌طور است.»

«تو هم همین احساس را داری؟»

گفتم: «البته.»

«ولی آکی اینطور نیست. وقتی تنها هستیم نمی‌خواهد به پیش برود.»

کمی طول کشید تا کلمات صحیح را انتخاب کنم. بالاخره گفتم: «این یک مسأله‌ی شخصی است.» «مردم روش‌های مختلفی برای رسیدن به خواسته‌هایشان دارند. کیتارو تو را خیلی دوست دارد -در این شکّی نیست- ولی رابطه‌ی شما آنقدر نزدیک و راحت است که او نمی‌تواند اوضاع را مانند اغلب مردم به پیش ببرد.»

«واقعاً اینطور فکر می‌کنی؟»

سرم را تکان دادم. «راستش را بخواهی، خودم هم واقعاً متوجّه نمی‌شوم. خودم هیچ وقت چنین تجربه‌ای نداشته‌ام. فقط می‌گویم که این هم امکان دارد.»

«گاهی به نظرم می‌رسد که هیچ میل جنسی‌ای به من ندارد.»

«مطمئنم که دارد. ولی بیان آن کمی باعث شرمساری‌اش می‌شود.»

«ولی ما بیست سالمان است، انسان‌های بالغی هستیم. آنقدر بالغ هستیم که شرمنده نشویم.»

گفتم: «بعضی افراد زودتر از دیگران بالغ می‌شوند.»

اریکا در مورد حرفم فکر کرد.به نظر می‌رسید از آن دسته آدم‌هایی باشد که همیشه با مشکلات مستقیماً مقابله می‌کنند.

ادامه دادم: «فکر می‌کنم کیتارو صادقانه به دنبال چیزی است.» «به روش خودش و با سرعت خودش. مسأله این است که فکر نمی‌کنم هنوز فهمیده باشد که چه می‌خواهد. به همین دلیل است که نمی‌تواند به پیش برود. اگر ندانی به دنبال چه هستی، به دنبال آن گشتن کار آسانی نیست.»

اریکا سرش را بلند کرد و به چشمان من خیره شد. شعله‌ی شمع در چشمانس انعکاس می‌یافت، یک نقطه‌ی کوچک و زیبای نورانی. آنقدر زیبا بود که مجبور شدم به سمت دیگری نگاه کنم.

اعتراف کردم: «البته تو او را بهتر از من می‌شناسی.»

دوباره آه کشید.

گفت: «در واقع، من پسرهای دیگری به جز آکی را هم می‌بینم.» «پسری در کلوپ تنیس هست که یک سال بزرگ‌تر از من است.»

این‌بار نوبت من بود که ساکت بمانم.

«من واقعاً آکی را دوست دارم و فکر نمی‌کنم که بتوانم این حس را در مورد شخص دیگری داشته‌باشم. هر وقت از او دورم، سینه‌ام درد می‌گیرد، همیشه یک جای مشخّص. راست می‌گویم. جایی در قلب من مخصوص اوست. ولی در عین حال، بخشی از وجودم \emph{اصرار می‌کند} که چیز جدیدی را امتحان کنم، که با مردم مختلف ارتباط برقرار کنم. می‌توانی اسمش را کنجکاوی بگذاری، عطشِ هرچه بیشتر دانستن. یک حس طبیعی است و هرچقدر هم که تلاش کنم، نمی‌توانم روی آن سرپوش بگذارم.»

یک گیاه سالم و سر حال را تصوّر کردم که دارد از اندازه‌ی گلدانش بزرگ‌تر می‌شود.

اریکا گفت: «وقتی می‌گویم سردرگم هستم، واقعاً همینطور است.»

گفتم: «پس باید به کیتارو هم بگویی که دقیقاً چه حسّی داری.» «اگر این مسأله که کسی دیگری را می‌بینی را از او مخفی کنی و او اتّفاقاً به موضوع پی ببرد، حسابی ضربه می‌خورد. تو که این را نمی‌خواهی.»

«ولی قبول می‌کند؟ این که با کس دیگری بیرون می‌روم؟»

گفتم: «به گمانم احساست را درک کند.»

«اینطور فکر می‌کنی؟»

گفتم: «بله.»

فکر کردم کیتارو متوجّه اعترافش بشود، چون خودش هم چنین احساسی داشت. یعنی، هر دو روی یک طول موج بودند. با این وجود، کاملاُ مطمئن نبودم که او کارهایی که اریکا می‌کرد (یا ممکن بود بکند) را با آرامش قبول کند. آدمی آنقدر قوی به نظرم نمی‌رسید. ولی اگر اریکا این راز را از او مخفی می‌کرد یا به او دروغ می‌گفت، اوضاع برایش بدتر می‌شد.

اریکا به شعله‌ی شمع که در نسیم دستگاه تهویه‌ی مطبوع تکان می‌خورد خیره شد. سپس گفت: «من اغلب یک خواب را می‌بینم.» «آکی و من در یک کشتی هستیم. سفری دور و دراز در یک کشتی بزرگ. با هم در یک کابین هستیم، شب و دیروقت است و از طریق پنجره‌ی کشتی می‌توانیم ماهِ کامل را ببینیم. ولی ماه از جنس یخ خالص و شفّاف است و نیمه‌ی پایینی آن در دریا غرق شده‌است. آکی به من می‌گوید: 'این شبیه ماه است.‍‍` 'ولی از جنس یخ است و فقط بیست سانتیمتر ضخامت دارد. یعنی وقتی که خورشید طلوع کند، آن را ذوب می‌کند. تا امکانش هست، باید خوب به آن نگاه کنی.` این رویا را بارها داشته‌ام. رویای زیبایی است. ماه هم همیشه همان است. همیشه بیست سانتیمتر ضخامت دارد. من به آکی تکیه داده‌ام، فقط ما دو نفر هستیم و بیرون کشتی امواج به آرامی حرکت می‌کنند. ولی هر بار که بیدار می‌شوم، به طرز غیر قابل تحمّلی ناراحت هستم.»

اریکا کوریتانی اندکی ساکت ماند. بعد دوباره شروع به صحبت کرد. «فکر می‌کنم که چقدر خوب می‌شد اگر آکی و من می‌توانستیم تا ابد به آن سفر ادامه دهیم. هر شب خود را در آغوش دیگری جمع می‌کردیم و از پنجره به ماهِ یخی خیره می‌شدیم. صبح که می‌آمد ماه ذوب می‌شد، و دوباره در شب ظاهر می‌گشت. ولی شاید موضوع این نباشد. شاید یک شب ماه در آسمان نباشد. حتّی فکر این هم مرا می‌ترساند. آنقدر می‌ترسم که حس می‌کنم بدنم به لرزه افتاده ‌است.»

وقتی روز بعد کیتارو را در کافی‌شاپ دیدم، از من پرسید که قرارمان چطور پیش رفت.

«بوسیدیش؟»

گفتم: «امکان ندارد.»

گفت: «نگران نباش -حتّی اگر او را می‌بوسیدی، وحشت نمی‌کردم.»

«چنین کاری نکردم.»

«دستش را هم نگرفتی؟»

«نه، دستش را هم نگرفتم.»

«پس چه کار کردید؟»

گفتم: «در سینما فیلم تماشا کردیم، قدم زدیم، شام خوردیم و صحبت کردیم.»

«همین؟»

«معمولاً در قرار اوّل خیلی تند نمی‌روند.»

کیتارو گفت: «واقعاً؟» «من هیچ وقت سر یک قرار معمولی نرفته‌ام، بنابراین نمی‌دانم.»

«ولی از معیّت او لذّت بردم. اگر دوست‌دختر من بود هیچ وقت نمی‌گذاشتم از جلوی چشمانم دور شود.»

کیتارو جملات من را مد نظر قرار داد. می‌خواست چیزی بگوید، ولی کمی بیشتر فکر کرد. بالاخره پرسید: «شام چه خوردید؟»

در مورد پیتزا و کیانتی برایش گفتم.

متعجّب شد: «پیتزا و کیانتی؟» «هیچ وقت نمی‌دانستم که پیتزا دوست دارد. ما همیشه به نودل‌فروشی‌ها و غذاخوری‌های ارزان رفته‌ایم. شراب؟ نمی‌دانستم که الکل هم می‌نوشد.»

کیتارو خودش هیچ وقت لب به الکل نمی‌زد.

«احتمالاً در مورد او چیزهایی هست که نمی‌دانی.» به همه‌ی سوالاتش در مورد قرارمان پاسخ دادم. در مورد فیلم وودی آلن (که کل داستان فیلم را برایش شرح دادم)، در مورد غذا (اینکه صورت‌حساب چقدر شد و دُنگی حساب کردیم یا نه)، در مورد لباسی که به تن داشت (لباس سفید کتانی، و موهایش را هم با سنجاق بسته‌بود) در مورد اینکه چه زیرپوشی به تن داشت (از کجا باید می‌دانستم؟) و در مورد موضوعاتی که صحبت کردیم. البته در مورد اینکه با شخص دیگری بیرون می‌رود چیزی به او نگفتم و به رویایش در مورد ماهِ یخی هم اشاره نکردم.

«تصمیم گرفتید که قرار بعدی کی باشد؟»

گفتم: «نه. تصمیم نگرفتیم.»

«چرا؟ مگر از او خوشت نیامد؟»

«دختر خیلی خوبی است. ولی اینطور که نمی‌شود. آخر او \emph{دوست‌دختر} توست، مگر نه؟ تو می‌گویی که عیبی ندارد ببوسمش، ولی هیچ رقمه نمی‌توانم این کار را بکنم.»

کیتارو کمی بیشتر فکر کرد. آخر سر گفت: «چیزی را می‌دانی؟» «از زمان اتمام مدرسه‌ی راهنمایی، پیش یک روانکاو می‌روم. والدین و معلّمانم همه پیشنهاد کردند که این کار را بکنم. چون گهگاه در مدرسه کارهایی می‌کردم. متوجّهی -نه از کارهای عادّی. ولی تا جایی که می‌دانم، جلسات روانکاوی کمکی نکرده است. در تئوری خوب است، ولی روانکاو کوچکترین اهمیّتی به من نمی‌دهد. طوری با تو رفتار می‌کنند که انگار کامل می‌دانند چه خبر است، بعد مجبورت می‌کنند که همینجور صحبت کنی. پسر، حتّی \emph{من} هم می‌توانم این کار را بکنم.»

«هنوز پیش روانکاو می‌روی؟»

«بله. دوبار در ماه. اگر از من بپرسی، پول دور ریختن است. اریکا در موردش برایت نگفت؟»

سرم را تکان دادن.

«راستش را بخواهی، نمی‌دانم چرا طرز تفکّرم اینقدر عجیب است. به نظر خودم، فقط دارم کارهای عادّی را به شیوه‌ای عادّی انجام می‌دهم. ولی مردم به من می‌گویند که تقریباً هر کاری که می‌کنم عجیب و غریب است.»

گفتم: «البته در مورد تو ویژگی‌هایی وجود دارد که به طور قطع عادّی نیست.»

«مثل چی؟»

«مثل صحبت با گویش کانسای.»

کیتارو اعتراف کرد: «شاید حق با تو باشد.» «این کمی غیر عادّی است.»

«آدم‌های عادّی تا این حد پیش نمی‌روند.»

«بله. احتمالاً حق با توست.»

«ولی تا جایی که می‌دانم، حتّی اگر کارهایت غیرعادّی هم باشد، باعث آزار و اذیّت کسی نمی‌شود.»

«فعلاً نمی‌شود.»

گفتم: «اشکال کار کجاست؟» آن زمان احتمالاً کمی ناراحت بودم (از دست چه کسی یا چه چیزی را نمی‌دانستم). می‌توانستم حس کنم که لحن صدایم کمی خشن می‌شود. «اگر کارهایت باعث آزار دیگران نمی‌شود، خب که چه؟ اگر دوست داری با گویش کانسای صحبت کنی، \emph{باید} صحبت کنی. همینطور ادامه بده. نمی‌خواهی برای امتحان ورودی درس بخوانی؟ خب نخوان. دوست نداری که دستت را زیر لباس‌های اریکا کوریتانی ببری؟ چه کسی گفته که باید این کار را بکنی؟ زندگیِ خودت است. باید هرکاری دوست داری بکنی و حرف دیگران را فراموش کنی.»

دهان کیتارو باز ماند و با تعجّب به من خیره شد. «چیزی را می‌دانی تانیمورا؟ تو پسر خوبی هستی. ولی گاهی \emph{زیادی} عادّی هستی؟ متوجّهی؟»

گفتم: «چکار می‌خواهی بکنی؟» «تو که نمی‌توانی شخصیّتت را عوض کنی.»

«دقیقاً. آدم نمی‌تواند شخصیّتش را عوض کند. من هم سعی دارم همین را بگویم.»

گفتم: «ولی اریکا دختر معرکه‌ای است.» «تو واقعاً برایش مهم هستی. هر کاری که می‌کنی، نگذار برود. دیگر هیچ وقت دختری به این معرکه‌ای پیدا نمی‌کنی.»

کیتارو گفت: «خودم می‌دانم. لازم نیست این‌ را به من بگویی.» «ولی دانستن این‌ها کمکی نمی‌کند.»

تقریباً دو هفته بعد، کیتارو از کار در کافی‌شاپ استعفا داد. می‌گویم استعفا داد، ولی در واقع، به طور ناگهانی دیگر سر کار حاضر نشد. با من تماس نگرفت، چیزی هم در مورد رفتن به مرخصی نگفت. تازه فصل شلوغی کار هم بود، بنابراین صاحب کافی‌شاپ حسابی از کوره در رفت. کیتارو حقوق یک هفته را طلبکار بود، ولی حتّی برای مطالبه‌ی آن هم باز نگشت. به همین راحتی ناپدید شد. باید بگویم که از دستش ناراحت شدم. فکر می‌کردم که برای هم دوستان خوبی هستیم و اینکه ناگهان با من قطع ارتباط کند برایم سخت بود. در توکیو دوست دیگری نداشتم.

دو روز آخرِ قبل از ناپدید شدنش، کیتارو خیلی ساکت بود. وقتی با او صحبت می‌کردم، زیاد حرف نمی‌زد. بعد هم رفت و ناپدید شد. می‌توانستم به اریکا کوریتانی زنگ بزنم و سراغش را از او بگیرم، ولی نتوانستم خودم را به این کار راضی کنم. فکر کردم هرچه بین آن دو اتّفاق افتاده به خودشان مربوط است، و کار درستی نیست که بیش از این خودم را درگیر ماجرا کنم. من هم باید کار خودم را در دنیای کوچکم به پیش می‌بردم.

بعد از این اتّفاقات، به دلایلی دائم به دوست‌دختر سابقم فکر می‌کردم. شاید احساسم به خاطر دیدن اریکا و کیتارو با هم بود. برایش نامه‌ی مفصّلی نوشتم و به خاطر رفتارم از او عذرخواهی کردم. می‌توانستم با او خیلی مهربان‌تر باشم. ولی هیچ وقت جواب نامه را دریافت نکردم.

اریکا کوریتانی را بلافاصله شناختم. او را فقط دو بار دیده بودم و از آن موقع شانزده سال گذشته‌بود. ولی امکان نداشت که در مورد او اشتباه کرده باشم. هنوز دوست‌داشتنی بود و شادابی و سرزندگی از چهره‌اش می‌بارید. لباس شب سیاه ابریشمی به تن داشت، با کفش‌های مشکیِ پاشنه‌بلند و دو ردیف گردنبند مروارید به دور گردن باریکش. او هم بلافاصله من را شناخت.  در یک مهمانی تست شراب در هتل آکاساکا\LTRfootnote{Akasaka} بودیم. یک مهمانی رسمی بود، به همین دلیل کت و شلوار و کراوات تیره پوشیده بودم. او نماینده‌ی یک شرکت تبلیغاتی بود، شرکتی که مسئول برگزاری مهمانی بود و به وضوح در اداره کردن آن سنگ‌تمام گذاشته بود. توضیح اینکه چرا من در آن مهمانی بودم مفصّل است.

پرسید: «آقای تانیمورا، چرا بعد از آن شبی که با هم بیرون رفتیم دیگر با من تماس نگرفتی؟» «امیدوار بودم که بتوانیم بیشتر با همدیگر صحبت کنیم.»

گفتم: «تو برای من بیش از حد زیبا بودی.»

لبخند زد. «از شنیدن این حرف خوشحالم، حتّی اگر به قصد چاپلوسی باشد.»

ولی آنچه که گفتم نه دروغ بود و نه چاپلوسی. بیش از حدّی که بتوانم جدّاً به او علاقه‌مند باشم زیبا بود. هم در گذشته و هم حالا.

گفت: «به کافی‌شاپی که در آن کار می‌کردی تلفن کردم، ولی گفتند که دیگر آنجا کار نمی‌کنی.»

بعد از اینکه کیتارو رفت، آن کار حوصله‌ام را سر می‌برد. دو هفته بعد استعفا دادم.

اریکا و من مختصراً زندگی‌مان در شانزده سال گذشته را مرور کردیم. بعد از کالج، من توسّط یک ناشر کوچک استخدام شده بودم، ولی بعد از سه سال استعفا داده و از آن موقع به عنوان نویسنده مشغول به کار بودم. در سن بیست و هفت سالگی ازدواج کرده بودم ولی هنوز بچّه نداشتم. اریکا هنوز مجرّد بود. به شوخی گفت: «سرِ کار آنقدر به من سخت می‌گیرند که وقتی برای ازدواج ندارم.» اوّل او بود که موضوع کیتارو را به میان آورد.

گفت: «آکی در دنْوِر\LTRfootnote{Denver} سرآشپزِ سوشی است.»

«دنور؟»

«دنور در کلورادو\LTRfootnote{Colorado}. حداقل با استناد به کارت‌پستالی که چند ماه قبل برایم فرستاد که اینطور است.»

«چرا دنور؟»

اریکا گفت: «نمی‌دانم.» «کارت‌پستال قبلی هم از سیاتل\LTRfootnote{Seattle} ارسال شده بود. آنجا هم سرآشپزِ سوشی بود. این مربوط به حدود یک سال پیش است. گهگاه برایم کارت پستال می‌فرستد. همیشه هم یک کارت احمقانه است که با عجله نوشته شده‌است. گاهی اوقات حتّی آدرس خودش را هم نمی‌نویسد.»

به تفکّر فرو رفتم: «سرآشپزِ سوشی». «آخرسر به کالج رفت؟»

سرش را تکان داد. «فکر می‌کنم اواخر همان تابستان بود که ناگهان اعلام کرد دیگر درس‌خواندن برای آزمون ورودی کافی است و به یک مدرسه‌ی آشپزی در اوساکا\LTRfootnote{Osaka} رفت. می‌گفت که واقعاً دوست‌دارد آشپزی کانسای را یاد بگیرد و به تماشای بازی‌ها در استادیوم کوشین\LTRfootnote{Koshien} برود، استادیوم ببرهای هنشین. البته از او پرسیدم 'چطور می‌توانی بدون صحبت با من در مورد موضوعی به این مهمی تصمیم بگیری؟ من چه می‌شوم؟`»

«و چه جوابی داد؟»

اریکا پاسخ نداد. فقط لب‌هایش را محکم به یکدیگر فشرد، انگار که اگر شروع به صحبت می‌کرد غرق در اشک می‌شد. سریعاً موضوع را عوض کردم.

«وقتی به آن رستوران ایتالیایی در شیبوبا رفتیم، یادم می‌آید که کیانتی ارزانی نوشیدیم. حالا نگاه کن، مشغول امتحان کردن شراب ناپا\LTRfootnote{Napa}ی اعلأ هستیم. سرنوشت عجیبی است.»

خودش را جمع کرد و گفت: «یادم می‌آید.» «یک فیلم وودی آلن هم تماشا کردیم. اسمش چه بود؟»

به او گفتم.

«فیلم قشنگی بود.»

موافق بودم. بدون شک یکی از شاهکارهای وودی آلن بود.

پرسیدم: «اوضاع با آن پسری که در کلوپ تنیس بود خوب پیش رفت؟»

سرش را تکان داد. «نه، آن طور که فکر می‌کردم با هم ارتباط برقرار نمی‌کردیم. شش ماه باهم بیرون می‌رفتیم، بعد هم از هم جدا شدیم.»

گفتم: «می‌توانم سوالی بپرسم؟» «البته کمی خصوصی است.»

«البته»

«نمی‌خواهم یک وقت به تو اهانت کرده باشم.»

«سعیم را می‌کنم.»

«با آن پسر خوابیدی، مگر نه؟»

اریکا با تعجّب به من نگاه کرد، گونه‌هایش سرخ شد.

«چرا الآن این موضوع را مطرح می‌کنی؟»

گفتم: «سوال خوبی است.» «مدّت‌ها بود که این موضوع برایم سوال بود. ولی سوال عجیبی پرسیدم. متأسفم.»

اریکا به آرامی سرش را تکان داد. «نه، مسأله‌ای نیست، ناراحت نشدم. فقط انتظارش را نداشتم. قضیه مربوط به خیلی وقت پیش است.»

به اتاق نگاه کردم. آدم‌ها با لباس‌های رسمی در آن پراکنده بودند. چوب‌پنبه‌ی شراب‌های گرا‌ن‌قیمت یکی پس از دیگری به بیرون می‌جهید. یک زنِ پیانیست داشت آهنگ 'مانند یک عاشق\LTRfootnote{Like Someone in Love}` را می‌نواخت.

اریگا گفت: «جواب سوالت مثبت است.» «با او چند بار رابطه داشتم.»

گفتم: «فقط کنجکاوی بود. عطشِ هرچه بیشتر دانستن.»

کمی لبخند زد. «درست است. کنجکاوی، عطشِ هرچه بیشتر دانستن.»

«اینگونه است که حلقه‌های رشدمان را توسعه می‌دهیم.»

گفت: «هرچه تو بگویی.»

«و حدس می‌زنم اوّلین‌ باری که با او رابطه داشتی، کمی بعد از قرارمان در شیبویا بود؟»

دفترچه‌ی خاطرات ذهنش را ورق زد. «اینطور فکر می‌کنم. حدوداً یک هفته بعد از آن بود. کل ماجرا را خوب یادم می‌آید. اوّلین بارم بود.»

به چشمانش نگاه کردم و گفتم: «و کیتارو موضوع را به خوبی درک کرد.»

سرش را پایین انداخت و انگشتانش را در میان مروارید‌های گردنبندش حرکت داد، انگار که بخواهد مطمئن شود همه‌ی آن‌ها هنوز سر جایشان هستند. آهِ کوتاهی کشید، شاید چیزی به یادش آمده بود. «بله، حق با توست. آکی قوّه‌ی ادراک خوبی داشت.»

«ولی اوضاع با آن پسر خوب پیش نرفت.»

سر تکان داد. «متأسفانه، من آنقدرها باهوش نیستم. باید از مسیر طولانی می‌رفتم. همیشه دور میدان می‌چرخم.»

\emph{
و این کاریست که همه‌ی ما انجام می‌دهیم: بی‌جهت از مسیر طولانی می‌رویم.
}
 می‌خواستم این را به او هم بگویم، ولی ساکت ماندم. به زبان نیاوردن چنین حقایقی، یکی دیگر از مشکلات عدیده‌ی من بود.
 
 «کیتارو ازدواج کرده‌است؟»
 
 اریکا گفت: «تا جایی که می‌دانم هنوز مجرّد است.» «حداقل اگر ازدواج کرده باشد هم به من نگفته است. شاید ما دو نفر از آن آدم‌هایی باشیم که هیچ وقت موفّق به ازدواج نشوند.»
 
 «شاید.»
 
 پرسیدم: «هنوز خواب آن ماهِ یخی را می‌بینی؟»
 
 سرش ناگهان به بالا پرید و به من خیره شد. با آرامش و آهسته، لبخندی روی صورتش پدیدار شد. یک لبخند کاملاً طبیعی و ناخودآگاه.»
 
 پرسید: «خواب من را به یاد داری؟»
 
 «به دلایلی بله.»
 
 «حتّی با وجود این که خواب یک نفر دیگر است؟»
 
 گفتم: «رویاها از آن چیزهایی هستند که می‌توانی از دیگران قرض بگیری و به دیگران قرض بدهی.»
 
 گفت: «نظر جالبی است.»
 
 کسی از پشت سر من اسمش را صدا کرد. وقتش بود که سر کار برگردد.
 
 همان طور که می‌رفت، گفت: «دیگر آن رویا را نمی‌بینم.» «ولی هنوز تک تک جزئیّات آن را به خاطر دارم. آنچه که می‌دیدم، آنچه که حس می‌کردم. این‌ها را نمی‌توانم فراموش کنم. و احتمالاً هیچ وقت هم فراموش نخواهم کرد.»
 
 وقتی در حال رانندگی هستم و رادیو آهنگ \emph{دیروز} را پخش می‌کند، نمی‌توانم به آن اشعار دیوانه‌وار که کیتارو در حمّام می‌خواند فکر نکنم. و پشیمانم که چرا شعر او را یادداشت نکرده‌ام. متن شعر او آنقدر عجیب بود که آن را تا مدّت‌ها به خاطر داشتم، ولی کم‌ کم از خاطرم محو شد، تا جایی که تقریباً همه‌ی آن را فراموش کردم. حالا فقط تکّه‌هایی از آن را به یاد دارم، و حتّی مطمئن نیستم که آنچه به خاطر دارم همان شعری باشد که کیتارو می‌خواند. با گذشت زمان، حافظه خواه ناخواه خودش را از نو می‌سازد.
 
 حدود بیست‌سالگی، چندین بار سعی کردم که خاطراتم را یادداشت کنم، ولی نتوانستم. آن زمان اطراف من آنقدر اتّفاق‌های متنوّع رخ می‌داد که خودم هم به سختی با آن‌ها پیش می‌رفتم، چه برسد به اینکه بخواهم صبر کنم و همه‌ی آن‌ها را در یک دفترچه یادداشت کنم. وبیشتر اتّفاقات هم آنطور نبود که پیش خودم فکر کنم، آه، باید این را یادداشت کرد.  تنها کاری که از دستم برمی‌آمد این بود که چشمانم را در مقابل بادی که از روبرو می‌وزید باز نگه دارم، نفسی تازه کنم و به پیش بروم.
 
 ولی خیلی عجیب است که کیتارو را خوب به یاد دارم. فقط چند ماه با هم دوست بودیم، با این وجود هر وقت به \emph{دیروز} گوش می‌کنم، صحنه‌ها و مکالمات او در ذهنم فوران می‌کند. صحبت‌کردن‌های ما دو نفر در حالی که او در وان حمّام خانه‌اش در دننچوفو خیس می‌خورد. صحبت کردن در مورد ترتیب ضربه زدن به توپ در تیم ببرهای هنشین، اینکه بعضی جنبه‌های برقراری رابطه تا چه حد می‌توانست مشکل ایجاد کند، اینکه درس‌خواند برای آزمون ورودی تا چه حد خسته‌کننده بود و اینگه گویش کانسای چقدر از نظر احساسی غنی بود. و آن قرار عجیب با اریکا کوریتانی هم به یادم می‌آید. و چیزهایی که اریکا -پشت میزِ روشن‌شده با نور شمع در آن رستوران ایتالیایی- اعتراف کرد. انگار که همه‌ی این اتّفاقات دیروز رخ داده‌اند. موسیقی این توانایی را دارد که خاطرات را زنده کند، گاهی آنقدر زنده که باعث رنجش آدم می‌شود.
 
 وقتی به خودم در سن بیست‌سالگی برمی‌گردم، بیشترِ آنچه به خاطر می‌آورم حس تنهایی است. دوست‌دختری نداشتم که جسم و جانم را گرم کند و دوستی هم نداشتم که با او درد دل کنم. هیچ نمی‌دانستم که روزها باید چه کنم و هیچ دیدی نسبت به آینده نداشتم. آن زمان بیشتر اوقات، در اعماق وجود خودم مخفی شده‌بودم. گاهی اوقات حتّی یک هفته‌ی کامل با کسی صحبت نمی‌کردم. این نوع زندگی تا یک سال ادامه یافت. واقعاً نمی‌دانم که این دوره از زندگی، زمستان سردی بود که حلقه‌ی رشد ارزشمندی از خود به جا گذاشت یا نه. آن زمان، من هم هر شب حس می‌کردم که دارم از پنجره به یک ماهِ یخی نگاه می‌کنم. یک ماه یخ‌زده‌ی شفّاف به ضخامت بیست سانتیمتر. ولی من این ماه را به تنهایی تماشا می‌کردم، بدون اینکه بتوانم سرمای زیبایش را با کسی به اشتراک بگذارم.
 
 «دیروز


دو روز قبل از فرداست،


روز بعد از دو روز پیش است.»
 
 


\end{document}